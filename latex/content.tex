%!TEX root = jt-thesis-main.tex

\section{English Abstract}
\begin{onehalfspace}
	
	The english abstract is required for all english theses at TU-Berlin.
\end{onehalfspace}
\clearpage

\section{Deutscher Abstract}
\begin{onehalfspace}
	
	Ein deutscher Abstract wird immer benötigt, für alle Abschlussarbeiten an der TU-Berlin. 
\end{onehalfspace}
\clearpage


\section{Introduction}
\label{sec:Introduction}
Gartner Inc. states that the number of interconnected devices will reach 20.4 billion by 2020 \footnote{Gartner Inc., R. (2017, February 7). Gartner Says 8.4 Billion Connected "Things" Will Be in Use in 2017, Up 31 Percent From 2016.Retrieved September 28, 2018, from https://www.gartner.com/en/newsroom/press-releases/2017-02-07-gartner-says-8-billion-connected-things-will-be-in-use-in-2017-up-31-percent-from-2016}.
To gather information from those devices, we need algorithms which efficiently sample and route data from sensor nodes (i.e., devices) to data sinks.
Energy expenditure will gain higher importance, especially for mobile devices such as battery-powered wearables and smartphones. 
A \ac{WSN} is often battery-powered and use cases span from home and health applications to the military sector~\cite{akyildiz2002wireless}. 
With low production costs of a node as a goal in \acp{WSN}~\cite{akyildiz2002wireless}, resource usage at the nodes is restricted. 
While reducing sampling frequencies leads to a higher life of a battery-powered sensor network, important changes in the observed phenomenon could be missed which reduces the quality of the data~\cite{akyildiz2002wireless}. 
A tradeoff between the quality of data and energy expenditure arises, thus an optimization is needed. \par
Different methods for optimizing \acp{WSN} were presented in surveys (e.g.,~\cite{abbasi2007survey},~\cite{sivrikaya2004time},~\cite{carrano2014survey}). 
The proposed areas of optimization include clustering algorithms~\cite{abbasi2007survey}, time synchronization~\cite{sivrikaya2004time}, duty cycling~\cite{carrano2014survey}, topology control~\cite{li2013survey}, in-network aggregation~\cite{fasolo2007network}, data compression~\cite{srisooksai2012practical}, and general routing techniques~\cite{al2004routing}~\cite{kulkarni2011particle}~\cite{singh2015survey}~\cite{rault2014energy}. 
TinyDB~\cite{madden2005tinydb}, ACQUIRE~\cite{sadagopan2003acquire} and COUGAR~\cite{yao2002cougar} introduce system architectures for query processing which consist of algorithms for sampling, and routing data requested by a user through a SQL-like language.


\subsection{Motivation}
\label{sec:motivation}

As stated above, surveys and taxonomies were presented for different areas of sensor networks. 
However, during our research, we did not encounter a survey or catalog focusing on sampling algorithms in sensor networks. 
This thesis aims to provide a catalog for a comprehensive overview of the existing sampling algorithms for the specific use case of sensor data. 
For researchers as well as practitioners, a collection of sampling algorithms is a valuable starting point for finding a solution for a design problem in a sensor network. \par
Thus, a taxonomy of the algorithms will be presented to provide a compact overview. 
Furthermore, algorithms which focus on areas other than sensing (like routing and topology building) in sensor networks will be presented. 
Combinations of said algorithms with sampling algorithms as well as combinations of sampling algorithms from different categories could inspire further research.


\subsection{Scientific Background}
\label{sec:Scientific Background}

Different types of \acp{SN}, like \acp{WSN}, \acp{RSN}, or wired \acp{SN} exist. 
While all of them are used to monitor a phenomenon, like the temperature in a room \footnote{Madden, S. et al. 2004. Intel Lab Data. [ONLINE] Available at: http://db.csail.mit.edu/labdata/labdata.html. [Accessed 29 November 2018].} or the behaviour of wildlife~\cite{bennett2011cranetracker} performance indicators vary in their significance. 
Intuitively, wired \acp{SN} do not consider energy expenditure as the primary concern. 
\acp{RSN} have a way to collect energy from the environment they are stationed at, through, e.g. solar and wind or more exotic variants like vibration \footnote{Perpetuum. 2018. Technology. 
[ONLINE] Available at: https://perpetuum.com/technology/. [Accessed 29 November 2018].}. 
To enable perpetual data collection in \acp{RSN}, techniques for allocating sensing tasks to sensors while handling the non linear emission of, e.g. solar and wind is crucial~\cite{liu2011perpetual}. 
\acp{WSN} often have a limited energy source at their disposal, making management of energy expenditure a top priority to increase network lifetime. 
As Cheng et al. state in their work~\cite{cheng2013stcdg} network lifetime is often defined in \acp{WSN} as the lifetime of the first node to run out of energy. \par
\acp{WSN} and \acp{RSN} are often designed with some nodes sensing the phenomenon and one or more sink nodes wich transfer the sensed data to a central base station. 
Based on the type of topology, e.g. a simple star structure\ref{fig:topologies}, sensed data is forwarded to the sink node directly, i.e. in a one-hop manner. 
More complex topologies, e.g. tree \ref{fig:topologies} or connected star structures, often require to relay sensed data through intermediary nodes to a sink node in a multi-hop fashion, as direct communication paths between sensing and sink nodes are not always possible~\cite{romer2004design}. \par
In more complex topologies with a lot of intermediary nodes, finding the optimal routing path from sensor node to the sink is not trivial. 
With increasing number of nodes, the complexity of the computation of the optimal route increases, as nodes have multiple neighbors from which to choose the next transmission step. 
Solving such a problem locally, i.e. nodes compute the next best step, with additional contraints, e.g. minimizing energy expenditure and meet quality of data thresholds can be an unfeasible task. 
On the other hand, outsourcing this task to the basestation could induce a major communication overhead with additional energy expenditure.
Techniques on topology building and route path finding will be discussed later in the thesis.

\begin{figure}%
    \centering
    \subfloat[Star Topology]{{\includegraphics[width=5cm]{images/topology-star-no-legend.jpg} }}%
    \qquad
    \subfloat[Tree Topology]{{\includegraphics[width=5cm]{images/topology-tree.jpg} }}%
    \caption{Example Topologies. Inspiration taken from Reina et al.~\cite{reina2013role}}%
    \label{fig:topologies}%
\end{figure}


\FloatBarrier


\subsection{Contributions}
\label{sec:contributions}

The contributions of this thesis go as follows: 
\begin{enumerate}
	\item A catalog of different sampling algorithms for sensor data gathering
	\item A taxonomy of those algorithms for a compact overview of the field of data sensing in \acp{SN}
	\item Combinations of different algorithms to provide a basis for further research
\end{enumerate}


\subsection{Thesis Outline}

\para{Section \ref{sec:Taxonomy}.}  In section \ref{sec:Taxonomy}, we present a high level taxonomy for classifying sampling algorithms. 
The section is split into four subsections. 
The first gives an overview of the described classes and subclasses and a motivation of why we chose such a partitioning. 
Each following section describes a class and its subclasses in detail. 
For each subclass, relevant information of multiple algorithms will be presented. 
We define relevant information as:
\begin{itemize}
	\item The problem(s) the algorithm tries to solve
	\item Basic workings of the algorithm
	\item Experimental results of the algorithm and how it compares to others
	\item Use case(s) (if any)
	\item Advantages and limitations of the algorithm the authors discuss
	\item Compatibility of the algorithms with other techniques
\end{itemize}   


\para{Section \ref{sec:Discussion}.} In section \ref{sec:Discussion}, we review algorithms not covered in the taxonomy from areas other than sampling, e.g. topology building and routing. 
This section is split into two subsections. 
The first subsection presents those algorithms and their relevant information. 
The definiton of relevant information for those algorithms does not change. 
In the second section we discuss possible combinations of different sampling algorithms with other sampling and non-sampling algorithms. 
The combinations found here could be used as a basis for further research.


\section{Taxonomy}
\label{sec:Taxonomy}

In this section we present our taxonomy of sampling algorithms for \acp{SN}.
We devided the mass of algorithms into three sections: Adaptive Sampling, Compressive Sampling and Data Sharing.
Many different terms for the concept of data retrieval from sensors exist in the literature. 
Data collection, data sampling, data gathering, and data sensing could be seen as synonyms, however, we found they are sometimes used in different contexts.
\par
Data collection is often mentioned in connection with the whole process of acquiring data through a sensors in a \acp{SN}.
% consider this: a query for temperature in a room is pushed down the network.
% a sensor in that room would power up and produce a stream of data...
% or the sensor node gets the instruction to produce data with the sensor e.g. every 3 sec for 18 sec
% so sensing would only apply to the process of getting digital data via a sensor while sampling could mean the same in context of sampling the "real values" of a happend event. Sampling could also mean getting a subset of data from multiple sensors.
A data stream produced by a sensor sensing a phenomenon is sampled by an sampling algorithm.
The data is then routed through the network to the destination.
The work of Yao et al.~\cite{yao2015edal} contributes, i.a., a technique, for finding the lowest cost path for sent data packages in a \acp{WSN} while additionally employing compressive sampling.
The authors use data collection to define this process.
Additionally, a survey by Di Francesco et al.~\cite{di2011data} defines the term data collection as the process of getting data from a sensor node to the sink node.
\par
Data gathering on the other hand could be used as a synonym for data collection.
Vukobratovic et al.~\cite{vukobratovic2010rateless} describe data gathering as the process of collecting sensed data from sensor nodes at specific sink nodes.
A sink node could be a basestation, e.g. a wired server, or another node in the \ac{SN} which acts as a cluster head.
Other works which use the term data gathering in the same context are often found in the domain of compressive sensing (e.g.~\cite{cheng2013stcdg},~\cite{luo2009compressive},~\cite{wang2012data})
\par
The definitons for data sampling and data sensing are also unclear in the literature. 
Zhao et al.~\cite{zhao2016cats} employ sampling in the context of sink nodes assigning sampling tasks, which consist of a time window and a sampling interval, to sensor nodes.
Trihinas et al.~\cite{trihinas2015adam} also do not differentiate between sampling and sensing.
Aquino et al.~\cite{aquino2014musa} on the other hand, discern sensing as the process of a preconfigured sensor unit measuring a physical phenomenon and sampling as the software taking samples form the data generated by the sensor unit.
The distinction is important as the authors point out, that generally, an online reconfiguration of the sensor sensing times is not possible.
Contrary, software is more flexible and sampling rates and intervals can be specified online.
\par
We will use data collection and data gathering as synonyms and differentiate between data sensing and data sampling.

% Sampling
	% Cats - Zhao = sampling interval and sampling tasks
		% a sampling task is alocated to a sensor node by a sink node
		% sampling task has a time window and a sampling interval
	% Adam - Trihinas = adaptive sampling techniques and sampling rate 
		% sampling == sensing data with sensors 
	% Musa - Aquio = a distinction is made between sampling and sensing
		% sensing = device (sensor) is configured to take regular samples over time
		% sampling = software takes samples from data produced by sensor
			% software is more flexible while sensor sensing generally cannot be reconfigured online

\subsection{Overview}
\label{sec:Overview}

% Picture of taxonomy at the beginning
% Explain picture and explain the terms
	% e.g. model based adaptive sampling is a prediction scheme ...
% Reasoning behind the partitioning of the algorithms
	% maybe if a assignment of a particular algorithm is not straightforward

\subsection{Adaptive Sampling}
\label{sec:Adaptive Sampling}

% All required definitions for adaptive sampling

\subsection{Compressive Sampling}


\subsection{Data Sharing}
\label{sec:Listings}


\section{Discussion}
\label{sec:Discussion}


\subsection{Other Algorithms}
\label{sec:Listings}

\subsubsection{Routing}
\label{sec:Listings}

\subsubsection{Topology Building}
\label{sec:Listings}


\subsection{Combination of Algorithms}
\label{sec:Listings}


\section{Conclusion}
\label{sec:Conclusion}


\section{References}
\label{sec:References}

